
\input{wsc15style.tex}     % download from author kit.  Style files for wsc formatting. Don't remove this line - required for generating the final paper!

\documentclass{wscpaperproc}
\usepackage{latexsym}
\usepackage{caption}
\usepackage{graphicx}
\usepackage{mathptmx}


%****************************************************************************
%% AUTHOR: You may want to use some of these packages. (Optional)
%\usepackage{amsmath}
%\usepackage{amsfonts}
%\usepackage{amssymb}
%\usepackage{amsbsy}
%\usepackage{amsthm}

\usepackage{xcolor}


\newcommand{\memo}[2]{\textcolor{#1}{#2}}
\newcommand{\added}[2]{\textcolor{#1}{#2}}
%\renewcommand{\memo}[2]{} % uncomment in the final version
%\renewcommand{\added}[2]{#2} % uncomment in the final version
\newcommand{\todo}[1]{\memo{red}{TODO: #1\\}}
\newcommand{\simon}[1]{\memo{green}{Simon: #1\\}}
\newcommand{\jm}[1]{\memo{blue}{JM: #1\\}}
\newcommand{\new}[1]{\added{orange}{#1}}

\begin{document}


\WSCpagesetup{Montanier, Carrignon, and Zerr}

\title{Model to study co-evolution of trade an culture}

\author{Jean-Marc Montanier\\ [12pt]
Barcelona Supercomputing Center\\
Carrer de Jordi Girona, 29, \\
08034 Barcelona, Spain\\
\and
Simon Carrignon\\ [12pt]
Barcelona Supercomputing Center\\
Carrer de Jordi Girona, 29, \\
08034 Barcelona, Spain\\
\and
Fabien Zerr\\ [12pt]
Alsace, Represent
}






\maketitle


\section*{ABSTRACT}

In this article our aim is to present a model suitable to test various hypothesis on economic and cultural co-evolution during the Roman Empire. The ultimate goal is to address debates from history of economy such as the type of economical market found in early empires. Agent based modelling is a good way to understand social self-organisation resulting from a large number of interactions. It is used in a wide range of fields of research, from economical market mechanisms to social dynamics. We therefore apply this type of framework to our study.

Here we present a model able it reproduces well known economical and cultural mechanism. Moreover this model can be envisioned to test hypothesis on those mechanisms. Given the appropriate match between the simulations and archaeological and historical data, the results of the model can be validated.


\section{INTRODUCTION}

%Roman empire and need of clean tool
The trade system of the Roman empire is generally considered to be the first complex European trade network. A variety of theories have been proposed to explain the organization and rise of this powerful trade network. However, without the appropriate sources and tools these theories are hard to falsify (garnsey et al 1983, lo cascio 2000). 

%type of question we want to answer to and why we need trade an culture
Among the question we would like to address is the on-going debate on the type of economic market in early empire~\cite{polanyi_trade_1957}. Two main view can be found in this debate. The first view state that the economical network of the Roman empire has evolve in a very different manner than our current trade network. The trade network is then perceived as qualitatively and quantitatively different. The second view state that the economical network of the Roman empire constitute a proto-version of our current trade network. The difference is then only quantitative but certainly not qualitative. We are specifically interested in the changes occurring in the production, trade and consumption of food. This can reflect the organization of the trade and cultural network. Moreover, this aspect has produced numerous historical records that we can now analyse.

%What is agent-based modeling and what can it do
Within the EPNet project~\cite{remesal_epnet_2014} we aim to create tools so that theories on the evolution of the trade network of the Roman empire can falsified . To do so we need tools that allow us to obtain the qualitative result of a specific theory. This qualitative result can then be compared to historical data and help to move forward. Following~\cite{epstein_growing_1996,lake_trends_2014,kohler_dynamics_2000,tesfatsion_agent-based_2003,epstein_why_2008} we consider Agent Based Modelling (ABM) to be a good way to test express theories in a falsifiable form. Within these models the social self-organisation of a society results from the interactions of a large number of agents. The behaviour of these agents is encoded so as to reflect the theory under test. It has been already used in a wide range of fields of research, from economical market mechanisms to the understanding of cultural goods production and exchange.

%What we want to bring to ABM with our work
The aim of this paper is to use ABM to model the evolution of culture and trade under a unique framework. Importantly we aim to create a framework with a low number of parameters in order to ensure that the dynamics observed are simple to analyse. In order to assess the quality of this framework we attempt to reproduce results previous obtained in other works study either economy or culture. We will then show the capacities of our framework to study the interaction between culture and trade.

%Organisation of the paper
\todo{write once we are more sure of what we put}
The paper is organised as follow

\section{PREVIOUS MODELISATIONS OF TRADE AND CULTURE}

This paper will provide a general view to the previous works done on the modelisation of trade, the modelisation of culture and the frameworks attempting to model both. While non-exhaustive this review aim to present the main trends on the previous work done on modelisation.

\subsection{Modelisation of Trade}
%expected utility: 
%	Shoemaker: classic model since second world war. Look at additions made to the model and purposes of the EU model (descriptive, predictive, postdictive and prescriptive). but failed partly at descriptive and predictive because do not take into account pshychological effects.
%	choice among risk options. maximisation of expected perceived utility accross all the options. Various perception of utility studied. Various modifications of probability also. define rational behaviour behaviour under risk.
%	descriptive: humans do not see problem as probabilities and do not reason on it
%	positivistic: can work in some very specific case such as billiard players. but in economy in most cases it does not work
%	failure of market on different aspects assumptions: noise, bias, irrational behaviour. can be patched up but it's better to look really at something new
%	has reveal that people precive and solve problems differently, and base of framework to discuss about these topics
%prospective theory:
%	Camerer : explain 10 economical trends that expected utility can not explain without making special cases. These are explained thanks to 3 effects from prospective theory: loss-aversion, reflection effects, non-linear weightening of probabilities
%		: loss aversion, people are much more sensitive to losses than to gains of the same magnitude
%		: reflection effects (reference dependence), measure of gain and loss from a reference point  people are (could have been present in EU)
%		: non-linear weightening of probabilities (diminishing sensitivity), bigger difference between 100 and 200 than between 1000 and 1100
%	Barberis : appear in 1979 and take the lead from EU
%	Argue that it's hard to know "how" to apply it
%	on top of the above effects speak also about probability weighting:
%		just transformation of probability, which is actually in EU. but here applied to cumulative probabilities. leads to overwheighting the tail
%	point to the difficulty to apply that
%ACE: 
%	tesfatsion: base of ACE -> goal is to simulate complex interactions from large number of agents in decentralised markets. and observe the macroeconomic regularities resulting from that. and feedback look. There is notably two fields that they are interested in, that are close to culture. These will particularly interestus
%    4 formation of economic network (extract cites) cite kirman_evolving_2001 tesfatsion_structure_2001
% 	 2 the evolution of behavioural norms (extract cites) cite epstein_learning_2001, axelrod_complexity_1997
%	gintis: particular work on pair exchage. Observe convergence to clearing prices and "optimal" utility.


\subsection{Modelisation of Culture}
% we also want to understand how culture will change in a setup. One interesting aspect in the changes of culture is to look at it from an evolutionary perspective. Especially useful to analyse variation in archeological records cite Lycett_2015 cite Henrich_evolution_2003. 
% Then whatever makes the culture (music listened, shapes of pottery ...) is copied by individuals. Each individual is eventually making small variations to it. and others will copy these small variations. What the field is studying then is how the selection mechanism oparate. What does make one prefer one thing from another. In the case of our study three main effects could be studied:
%
%variation:essential : o'brien_variation_1990. may occure by copy error Schillinger et al 2014a, but can also happen intentionally ziman_technological_2000

%
%transmission: This further involve social learning, a topic in itself cite Heyes_social_1994. This mechanism does not necessarily have to be teaching. Can range from stimulus to emulation passing by imitation. 
%
%differential replication: how comes that one idea is more reproduced than the others. can be of different type: directional, disruptive or stabilization. thre are a range of biases that have been found in this selection process. Rendell_cognitive_2011 It is this process that will interest us more because it's simplier to test
%
%random copy: it is the baseline assumption. actually that is a good one to predict tastes in music. 
%frequency dependent: that's one where we consider that the most common will get even more common (conformity) or that the less common will be prefered (rarity)
%model biased - prestige: as the title say it, that's studying how the prestige of one person (kinda the boss) can bias the others in following her/him.
%


\subsection{Modelisation of Trade and Culture}
%stuff with both in it
%macmillan et bentley


\section{MODEL DESCRIPTION}

\subsection{Agent-based consumers and traders}
%tell the story of what is happening

%%%%%
%%TODO: Find refs
%%
In the simulation agents produce goods that they will barter (cf section \ref{trade}) and consume (cf section \ref{consumption}). This goods production should reflect the historical good production of the period studied and agent could produce various good in order to barter or sell them using the trade network.

%%%%%
%%TODO: Find refs
%%
During the simulation the agents ``consume'' the good. This consumption could be vital (if agent do consume certain good they die) or not (as in \cite{macmillan_agent-based_2008}), it could be set as fixed, environmental constraint shared by everyone (the ``intrinsic'' value of things) or could be subject to historical or geographical cultural ``taste'' and variations.

Using the network, the agents are able to barter/trade/exchange resources altogether. 



\subsection{Model structure} 
\begin{itemize}
	\item Resources :\\
		There are $n$ kind of resources, each one is produced by one or more Agents. Agents can exchange those resources and consume it following their need.
	\item  Agents :\\
		The whole population is defined by a vector of $m$ agents: 
		$$ population= (a_i, \cdots ,a_m) $$
		Each agent $i$ is defined by the following vectors:
		\begin{itemize}
			\item Price $(p^i_1,\cdots,p^i_n)$ : an amount of money that could be use when in exchange of a good.
			\item Quantity $(q^i_1,\cdots,q^i_n)$ : the quantity of each food owned by each agent.
			\item Subjective value $(u^i_1,\cdots,u^i_n)$ : the values that each agent associates with each resource.
			\item Need $(b^i_1, \cdots, b^i_n)$ :  the quantity of each resource that each agent need to ``survive''. In the first experiments it only says the amount of resource that the agent will consume. It can be seen as the ``intrinsic'' value of the good.
		\end{itemize}
\end{itemize}

\todo{find another title}
\subsection{Effects tested}

\subsubsection{Good Production}\label{production}
The production methods can vary depending of geographical or historical factors. In order to validate the relevance of the model, we will follow \cite{gintis_emergence_2006}. In its model each agent produces one kind of good in a given and finished quantity which is $n$, the number of different resource available in a simulation, thus allowing him to be able to trade every other goods that it does not produce.

\subsubsection{Good Consumption}\label{consumption}
Depending on the kind of good agents consume (if they are ``vital'' or not\ldots) it allow us to test a wide range of hypothesis, from purely economical one to cultural consumption assumptions.  In an evolutionary perspective this consumption function could change trough time and is a good way to score the agent fitness and to know which strategies (the list of prices) are the best.

In the current version of the model, and following
\cite{gintis_emergence_2006} to be sure that we could reproduce the expected mechanism, all the goods are consumed in the same proportion by all the agents during all the simulation. A fixed-for-everyone proportion $b_1,\cdots,b_n$ of each resource $(1, \cdots, n)$ is given at the beginning of the simulation.

\subsubsection{Cultural Evolution}
In our model the ``cultural change'' is seen as the variation during time of the space of belief of all the agents. Pragmatically it's represented by variation in distribution of subjective value vector.

In the first experiment we will try simple random copying and frequency-dependent copying as \cite{mesoudi_random_2009} to show the usefulness of the method to study cultural change. 


\subsubsection{Economical Trading}\label{trade}
Here again, the idea is to propose a model in which several different bargaining/trading mechanisms can be tested. 

But to begin, we will implement a bargain mechanism already implemented in an Agent Based Model which is the one done by
\cite{gintis_emergence_2006} in order to be quickly able to compare our results and validate ability of the model to recreate economical behaviors. 

In the Ginti's model that we are following here, the goal is only to see if a market equilibrium can be reach in a bartering decentralized economy model. So the cognitive trading abilities of the agents are set at their minimum. 

During the barter process, all agents meet all the producers of all goods and choose to exchange a certain amount of the good it produce with a certain amount of the other good depending on their own stock and the value they assign respectively to the resource it produce and the resources needed, in a totally rational way, without any cognitive bias.

\section{EXPERIMENTAL SETUP}

\section{RESULTS}

\section{FUTURE WORK}
\subsection{Good Production}
Those production could change during the simulation based on meteorological data or historical knowledge and reflect some complex economical specialisation mechanisms \cite{bentley_specialisation_2005}.


\subsection{Cultural Evolution}
This subjective value is the value that agents ``put'' on each resources. It could be learn using any mechanism of ``social learning'' (as defined by \cite{lycett_cultural_2015}) known in the literature (teaching, different kind of copying mechanisms,\ldots) and/or integrate any cognitive/environmental bias that could be studied (see again \cite{lycett_cultural_2015} for some kind of bias that could be implemented).


\subsection{Good Consumption and economical trading}
%ACE and prospective trading
%\subsection{Economical Trading}
Later it could be used to study other model such as those developed by 
\cite{rubinstein_equilibrium_1985} and to test more general proposal, as those done by \cite{polanyi_trade_1957,polanyi_livelihood_1977} about the early economical mechanisms.

Moreover, and in order to be more realistic, and consistently with the cognitive and agent based approach of the project, we will also test assumptions made by people from \emph{Prospect Theory} as proposed by 
\cite{kahneman_prospect_1979}, see also
\cite{camerer_prospect_2004}. The idea here is that traditional economical studies use formal mathematical models that suppose people acting as rational agent, which is a presumption that is far from being valid  in every day life. 

Top palliate this problem, Prospect Theory put the focus on the study of the human cognitive abilities and the impact those abilities brings on economics issues. Among various aspect on which prospect theory shed light one can look at how cognitive abilities modify the decisions-making process of human under risk
\cite{weber_disposition_1998}.

Moreover, we think that this approach fit perfectly with our agent based modeling approach (also called Agent based Computational Economy, ACE, when dealing with economical problems, see
\cite{tesfatsion_introduction_2001}).


\section{CONCLUSION}




\bibliographystyle{wsc}
\bibliography{wsc.bib}  
\end{document}

