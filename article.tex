
\input{wsc15style.tex}     % download from author kit.  Style files for wsc formatting. Don't remove this line - required for generating the final paper!

\documentclass{wscpaperproc}
\usepackage{latexsym}
\usepackage{caption}
\usepackage{graphicx}
\usepackage{mathptmx}


%****************************************************************************
%% AUTHOR: You may want to use some of these packages. (Optional)
%\usepackage{amsmath}
%\usepackage{amsfonts}
%\usepackage{amssymb}
%\usepackage{amsbsy}
%\usepackage{amsthm}

\usepackage{xcolor}


\newcommand{\memo}[2]{\textcolor{#1}{#2}}
\newcommand{\added}[2]{\textcolor{#1}{#2}}
%\renewcommand{\memo}[2]{} % uncomment in the final version
%\renewcommand{\added}[2]{#2} % uncomment in the final version
\newcommand{\todo}[1]{\memo{red}{TODO: #1\\}}
\newcommand{\simon}[1]{\memo{green}{Simon: #1\\}}
\newcommand{\jm}[1]{\memo{blue}{JM: #1\\}}
\newcommand{\new}[1]{\added{orange}{#1}}

\begin{document}


\WSCpagesetup{Montanier, Carrignon, and Zerr}

\title{Model to study co-evolution of trade an culture}

\author{Jean-Marc Montanier\\ [12pt]
Barcelona Supercomputing Center\\
Carrer de Jordi Girona, 29, \\
08034 Barcelona, Spain\\
\and
Simon Carrignon\\ [12pt]
Barcelona Supercomputing Center\\
Carrer de Jordi Girona, 29, \\
08034 Barcelona, Spain\\
\and
Fabien Zerr\\ [12pt]
Alsace, Represent
}






\maketitle


\section*{ABSTRACT}

\todo{Rewrite later}
In this article our aim is to present a model suitable to test various hypothesis on economic and cultural co-evolution during the Roman Empire. The ultimate goal is to address debates from history of economy such as the type of economical market found in early empires. Agent based modelling is a good way to understand social self-organisation resulting from a large number of interactions. It is used in a wide range of fields of research, from economical market mechanisms to social dynamics. We therefore apply this type of framework to our study.

Here we present a model able it reproduces well known economical and cultural mechanism. Moreover this model can be envisioned to test hypothesis on those mechanisms. Given the appropriate match between the simulations and archaeological and historical data, the results of the model can be validated.


\section{INTRODUCTION}

%Roman empire and need of clean tool
\todo{don't we have a more recent and better ref for that ?}
The trade system of the Roman empire is generally considered to be the first complex European trade network. A variety of theories have been proposed to explain the development of this network as well as its organization. However, the appropriate tools are currently missing to falsify these theories  \cite{garnsey_trade_1983}. 

%type of question we want to answer to and why we need trade an culture
We want notably to address the on-going debate about the characteristics of the economic market in Roman empire~\cite{polanyi_trade_1957}. Two main views are found on this debate. The first view states that the economical network of the Roman empire had characteristics very different than our modern trade network. The trade network of the Roman empire is then perceived to be qualitatively and quantitatively different than our modern trade network. The second view state that the economical network of the Roman empire constitutes a proto-version of our modern trade network. The difference between the two networks is then only quantitative and not qualitative. 
\jm{not sure that food item is the right word}
Among all the goods transiting on the Roman network, we will focus on the food items. 
\jm{i find the justification weak. any ideas ?}
These are interesting as they reflect the organization of both the trade network and the cultural network. Moreover, the food items were circulating in amphorae which arrived up to us. As a consequence it will be possible to compare the qualitative output of a simulation with the qualitative structure of the historical data.

%What is agent-based modeling and what can it do
\jm{maybe this paragraph should go before ?}
Within the EPNet project~\cite{remesal_epnet_2014} we aim to create tools able to falsify theories on the characteristics of the trade network of the Roman empire. These tools have to produce the qualitative characteristics resulting from a specific theory. This qualitative result can then be compared to historical data and highlight the differences between the reality and the simulations. Following~\cite{epstein_growing_1996,lake_trends_2014,kohler_dynamics_2000,tesfatsion_agent-based_2003,epstein_why_2008} we consider Agent Based Modelling (ABM) is a good framework to express theories in a falsifiable form. Within these models the social self-organisation of a society results from the interactions of a large number of agents. The behaviour of these agents is encoded so as to reflect the theory under test. It has been already used in a wide range of fields of research, from economical market mechanisms to the understanding of cultural goods production and exchange.

%What we want to bring to ABM with our work
\todo{the justification of reproducing previous works should be stronger}
The aim of this paper is to develop an ABM to model the evolution of culture and trade. Importantly we aim to create a model with a low number of parameters in order to ensure that the dynamics observed are simple to analyse. In order to assess the quality of this model we use it to reproduce qualitatively certain dynamics observed in either economy or culture. We will then show the capacities of our model to study the interaction between culture and trade.

%Organisation of the paper
\todo{write once we are more sure of what we put}
The paper is organised as follow

\section{PREVIOUS MODELISATIONS OF TRADE AND CULTURE}

\todo{and/or is ugly, look for another way}
This section will provide a brief review of the previous approaches to model trade and/or culture. While non-exhaustive, this review presents the mains trends of the previous work done.

\subsection{Modelisation of Trade}

Multiple approaches have been proposed to model the economical interactions between humans. Among them, this section will review three which constitute major trends and are the most relevant to our study. The approaches are presented in chronological order.

The first approach to be reviewed is the Expected Utility (EU) which became a main trend after the second world war~\cite{schoemaker_expected_1982}. The goal of this model was to propose a first explanation to the economical actions of humans along the following axes: descriptive, predictive, postdictive and prescriptive. More precisely this approach model individuals as agents facing a choice among risky options. Each agent is aware of the probability that an event will happen and the utility (economical reward) associated to that event. Each agent can then maximise its expected utility by choosing the proper actions. By applying this methodology to various economical problems it became clear that humans do not see problems as probabilities and do not directly reason on them. From the perspective of this theory, humans show a collection of irrational behaviours, notably a strong aversion to risk.

In an attempt to overcome the shortcomings of the EU models a new theory was introduced, the prospective theory. This theory conserve an approach close to the EU: individuals are modelled as agents facing risky actions each with a certain probability and utility. However, the agents do not face these problems by maximising the expected utility. They are rather endowed with behaviours composed of three key elements~\cite{camerer_prospect_2004}. First, the ``loss aversion'' element models the fact that individuals are much more sensitive to loss than to gains of the same magnitude. Second, the ``reference dependence'' models the fact that individuals measure their losses and gains with regards to their current possessions. Finally, individuals do not weight linearly the probabilities e.g. an individual will perceive a higher gain between 100 and 200 than between 1000 and 1100. Multiple difficulties have been encountered in applying this approach to practical problems of the real life. These are attributed to the lack of definition of certain elements of the theory and therefore the near impossibility to measure them~\cite{barberis_thirty_2012}.

The two previous approaches address an economical problem as a block which all agents face in a similar fashion. As a consequence, the emerging properties due to the interaction of numerous agents can not be studied. Moreover, these approaches are not suitable to model systems composed of individuals with various characteristics and interests. The Agent-based Computational Economics (ACE) fields propose specifically to model the complex outcomes raising from a large number of interactions in a decentralized market~\cite{tesfatsion_agent-based_2003}. Naturally this approach is interesting for our project based on multi-agent methodology and it constitutes a source of hypothesis to test to characterise the economical network of the Roman empire. More specifically, multiple hypothesis are proposed on the formation of different types of economical networks depending on the economical rules between the agents~\cite{kirman_evolving_2001,tesfatsion_structure_2001}. Another interesting track is the study of the evolution of behavioural norms in an economical network~\cite{epstein_learning_2001,axelrod_complexity_1997}. Finally~\cite{gintis_emergence_2006} has proposed a specific analysis on the emergence of price structure when only pair exchanges are considered. 
\todo{put the section}
This latest work constitutes naturally a strong reference and will be explained in further details in section XXX.


\subsection{Modelisation of Culture}

The second aspect we wish to model is the evolution of the culture in the Roman empire. The evolutionary framework (generally used in Biology) has been proposed to study the changes of culture and found to be suitable,  particularly in the case of historical and archaeological studies~\cite{lycett_cultural_2015,henrich_evolution_2003}.

Within this framework, the individual preferences for cultural traits (such as music taste or shapes of pottery) are copied from one individual to the other. Each individual is then adding some small variations to this cultural trait she just acquired. The individual will in turn continue to spread these cultural traits which will be further adopted by other individuals. Studies are then performed on the three key steps of the framework: the transmission, the variation and the differential replication. Interestingly the study of these elements is not only done thanks to humans but also thanks to other animals.

The transmission mechanisms are the means used to transfer cultural traits from one individuals to another. A large number of social learning schemes can be envisioned~\cite{heyes_social_1994}. Notably, the mechanisms do not necessarily have to be as complex as teaching, since emulation and imitation can be sufficient to pass on certain cultural traits.

The variation mechanisms are at the heart of the creation of the creation of new cultural elements~\cite{obrien_variation_1990}. These may happen due to some errors during the copy process~\cite{schillinger_copying_2014}. In more recent culture we can witness the apparition of mutations with intentionality~\cite{ziman_technological_2003}.

The last element is the one that will interest us the most in this study: the differential replication between different cultural traits. If a cultural trait is found more often than the others in the population it means that something modified the rate at which it is adopted by each agent. 
%The reproduction rate of ideas in a population can be characterise by different rates: directional, disruptive or stabilisation~\cite{rendell_cognitive_2011}. 
Multiple processes can be at play behind the reproduction of cultural traits. Among them the three following will interest us due to their simplicity to test in an agent-based modelisation. First is the random copy of one trait from an agent to another agent. 
\todo{get a ref on that}
This type of mechanism reproduces well the changes in musical preferences. Second are the the copy based on the frequency at which a cultural trait is observed. If a common trait is reproduced more often then we speak about ``conformity.'' If on the contrary the less common trait is preferred for reproduction we speak of ``rarity''. Finally a trait can be preferred depending on whom is carrying it. Among these preferences the prestige of the individual carrying it can constitute an explanation for the adoption foreign cultural traits in the Roman empire.


\subsection{Modelisation of Trade and Culture}

To the best of our knowledge two works have previously proposed frameworks that could be used to model the co-evolution of trade and culture. The model in~\cite{bentley_specialisation_2005} is proposed to study the wealth inequalities in a dynamic network. In this model the agents are in a virtual environment and can produce, consume and trade two products. This model has two major constraints which makes it unfit for the study presented here. First, the products are only for trade purposes and are not used as cultural factors, i.e. some goods can not be accumulated so that the agent show them to others. The cultural aspects appear only through the price associated to each product. Second, only two goods are considered and the system can not scale up to an undetermined number of products without major modifications.
% Agents can negotiate the rate at which the two products will be traded. The agents have a will (hard-coded in their behaviour) to keep their wealth and number of agents within their trading network above a threshold.

The model proposed in~\cite{macmillan_agent-based_2008} studies the settlement of primitive agricultural societies. In this model agents produce, trade and consume 2 essentials goods and 2 non-essentials goods. This model is interesting since it introduce the notion of non-essential goods which can be used to reflect the culture of a society. However, this framework is constrained as it does not allow the evolution of the trading network. Moreover the behaviour chosen for the agents is rather complex and not justified by observations made in economy or evolution of culture.

The model presented in this work proposes to integrate the most interesting features from~\cite{bentley_specialisation_2005} and~\cite{macmillan_agent-based_2008}.


\section{MODEL DESCRIPTION}

\todo{find another title}
\subsection{Agent-based consumers and traders}
%tell the story of what is happening

This model is composed of multiple agents producing goods of a fixed number of types. The agents are endowed with storage abilities for each good that can be produced. On top of	producing the agents have the ability to exchange and consume the goods. The specific rules for production, consumption and trading will vary depending on the hypothesis tested. Nevertheless the structure of the environment and the proceeding of a simulation will be the same for all hypothesis tested in this model.

One time step of the simulation is divided in three phases. In the first phase all agents produce goods. Each agent is assigned one type of good, and it will produce only this type of good at each first phase. In the second phase all agents enter trade with other agents according to their own interest. Each agent has the possibility to choose with which other agent it will trade products. Once two agents agree to trade, the agents decide which products will be exchanged and in which quantities. When an agreement is found, the trade occur and the agents look for other potential partners. Within one trading phase, an agent can meet only one time each other agents. In the third phase all agent consume goods according to their own rules. A unit of a good consumed disappears from the stock of the agent.

%In the simulation agents produce goods that they will barter (cf section \ref{trade}) and consume (cf section \ref{consumption}). This goods production should reflect the historical good production of the period studied and agent could produce various good in order to barter or sell them using the trade network.

More formally the model is composed of the following elements

\begin{itemize}
	\item Resources :\\
		There are $n$ kind of resources, each one is produced by one or more Agents. Agents can exchange those resources and consume it following their need.
	\item  Agents :\\
		The whole population is defined by a vector of $m$ agents: 
		$$ population= (a_i, \cdots ,a_m) $$
		Each agent $i$ is defined by the following vectors:
		\begin{itemize}
			\item Price $(p^i_1,\cdots,p^i_n)$ : an amount of money that could be use when in exchange of a good.
			\item Quantity $(q^i_1,\cdots,q^i_n)$ : the quantity of each food owned by each agent.
			\item Subjective value $(u^i_1,\cdots,u^i_n)$ : the values that each agent associates with each resource.
			\item Need $(b^i_1, \cdots, b^i_n)$ :  the quantity of each resource that each agent need to ``survive''. In the first experiments it only says the amount of resource that the agent will consume. It can be seen as the ``intrinsic'' value of the good.
		\end{itemize}
\end{itemize}

\todo{find another title}
\subsection{Effects tested}

\subsubsection{Good Production}\label{production}
The production methods can vary depending of geographical or historical factors. In order to validate the relevance of the model, we will follow \cite{gintis_emergence_2006}. In its model each agent produces one kind of good in a given and finished quantity which is $n$, the number of different resource available in a simulation, thus allowing him to be able to trade every other goods that it does not produce.

\subsubsection{Good Consumption}\label{consumption}
Depending on the kind of good agents consume (if they are ``vital'' or not\ldots) it allow us to test a wide range of hypothesis, from purely economical one to cultural consumption assumptions.  In an evolutionary perspective this consumption function could change trough time and is a good way to score the agent fitness and to know which strategies (the list of prices) are the best.

In the current version of the model, and following
\cite{gintis_emergence_2006} to be sure that we could reproduce the expected mechanism, all the goods are consumed in the same proportion by all the agents during all the simulation. A fixed-for-everyone proportion $b_1,\cdots,b_n$ of each resource $(1, \cdots, n)$ is given at the beginning of the simulation.

\subsubsection{Cultural Evolution}
In our model the ``cultural change'' is seen as the variation during time of the space of belief of all the agents. Pragmatically it's represented by variation in distribution of subjective value vector.

In the first experiment we will try simple random copying and frequency-dependent copying as \cite{mesoudi_random_2009} to show the usefulness of the method to study cultural change. 


\subsubsection{Economical Trading}\label{trade}
Here again, the idea is to propose a model in which several different bargaining/trading mechanisms can be tested. 

But to begin, we will implement a bargain mechanism already implemented in an Agent Based Model which is the one done by
\cite{gintis_emergence_2006} in order to be quickly able to compare our results and validate ability of the model to recreate economical behaviors. 

In the Ginti's model that we are following here, the goal is only to see if a market equilibrium can be reach in a bartering decentralized economy model. So the cognitive trading abilities of the agents are set at their minimum. 

During the barter process, all agents meet all the producers of all goods and choose to exchange a certain amount of the good it produce with a certain amount of the other good depending on their own stock and the value they assign respectively to the resource it produce and the resources needed, in a totally rational way, without any cognitive bias.

\section{EXPERIMENTAL SETUP}

\section{RESULTS}

\section{FUTURE WORK}
\subsection{Good Production}

\todo{speak about good collection from the environment}

Those production could change during the simulation based on meteorological data or historical knowledge and reflect some complex economical specialisation mechanisms \cite{bentley_specialisation_2005}.


\subsection{Cultural Evolution}
This subjective value is the value that agents ``put'' on each resources. It could be learn using any mechanism of ``social learning'' (as defined by \cite{lycett_cultural_2015}) known in the literature (teaching, different kind of copying mechanisms,\ldots) and/or integrate any cognitive/environmental bias that could be studied (see again \cite{lycett_cultural_2015} for some kind of bias that could be implemented).


\subsection{Good Consumption and economical trading}

\todo{speak about death of the agents}

%ACE and prospective trading
%\subsection{Economical Trading}
Later it could be used to study other model such as those developed by 
\cite{rubinstein_equilibrium_1985} and to test more general proposal, as those done by \cite{polanyi_trade_1957,polanyi_livelihood_1977} about the early economical mechanisms.

Moreover, and in order to be more realistic, and consistently with the cognitive and agent based approach of the project, we will also test assumptions made by people from \emph{Prospect Theory} as proposed by 
\cite{kahneman_prospect_1979}, see also
\cite{camerer_prospect_2004}. The idea here is that traditional economical studies use formal mathematical models that suppose people acting as rational agent, which is a presumption that is far from being valid  in every day life. 

Top palliate this problem, Prospect Theory put the focus on the study of the human cognitive abilities and the impact those abilities brings on economics issues. Among various aspect on which prospect theory shed light one can look at how cognitive abilities modify the decisions-making process of human under risk
\cite{weber_disposition_1998}.

Moreover, we think that this approach fit perfectly with our agent based modeling approach (also called Agent based Computational Economy, ACE, when dealing with economical problems, see
\cite{tesfatsion_introduction_2001}).


\section{CONCLUSION}




\bibliographystyle{wsc}
\bibliography{wsc.bib}  
\end{document}

