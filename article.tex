
\input{wsc15style.tex}     % download from author kit.  Style files for wsc formatting. Don't remove this line - required for generating the final paper!

\documentclass{wscpaperproc}
\usepackage{latexsym}
\usepackage{caption}
\usepackage{graphicx}
\usepackage{mathptmx}
\usepackage[utf8]{inputenc}

%****************************************************************************
%% AUTHOR: You may want to use some of these packages. (Optional)
%\usepackage{amsmath}
%\usepackage{amsfonts}
%\usepackage{amssymb}
%\usepackage{amsbsy}
%\usepackage{amsthm}

\usepackage{mathtools}
\DeclarePairedDelimiter\abs{\lvert}{\rvert}%
\DeclarePairedDelimiter\norm{\lVert}{\rVert}%

\makeatletter
\let\oldabs\abs
\def\abs{\@ifstar{\oldabs}{\oldabs*}}
\let\oldnorm\norm
\def\norm{\@ifstar{\oldnorm}{\oldnorm*}}
\makeatother

\usepackage{xcolor}
    \usepackage{cite}


\newcommand{\memo}[2]{\textcolor{#1}{#2}}
\newcommand{\added}[2]{\textcolor{#1}{#2}}
%\renewcommand{\memo}[2]{} % uncomment in the final version
%\renewcommand{\added}[2]{#2} % uncomment in the final version
\newcommand{\todo}[1]{\memo{red}{TODO: #1\\}}
\newcommand{\simon}[1]{\memo{green}{Simon: #1\\}}
\newcommand{\jm}[1]{\memo{blue}{JM: #1\\}}
\newcommand{\xrc}[1]{\memo{orange}{XRC: #1\\}}
\newcommand{\new}[1]{\added{orange}{#1}}

\begin{document}

\WSCpagesetup{Montanier, Carrignon, Rubio and Zerr}

\title{Model to study co-evolution of trade an culture}
\maketitle

\begin{figure*}[htb]
{
\centering
Jean-Marc Montanier\\
Simon Carrignon\\ 
Xavier Rubio-Campillo\\
\vspace{12pt}
Barcelona Supercomputing Center\\
Carrer de Jordi Girona, 29, \\
08034 Barcelona, Spain\\
}
\end{figure*}







\section*{ABSTRACT}

\todo{Rewrite later}
In this article our aim is to present a model suitable to test various hypotheses on economic and cultural co-evolution during the Roman Empire. The ultimate goal is to address debates from history of economy such as the type of economical market found in early empires. Agent based modelling is a good way to understand social self-organisation resulting from a large number of interactions. It is used in a wide range of fields of research, from economical market mechanisms to social dynamics. We therefore apply this type of framework to our study.

Here we present a model able it reproduces well known economical and cultural mechanism. Moreover this model can be envisioned to test hypothesis on those mechanisms. Given the appropriate match between the simulations and archaeological and historical data, the results of the model can be validated.


\section{INTRODUCTION}\label{sec:intro}

%XRC trade networks are increasingly being recognised as a key process in ancient Mediterranean
Trades are being increasingly recognised as a key process in the structure of ancient Mediterranean societies. For example, the use of table ware in the Roman empire  has been linked to the trade network~\cite{brughmans_connecting_2010}. Interestingly this trade network has been shown to be a market economy~\cite{temin_market_2001}. The work presented in this paper proposes a framework to further study the relation between trade and culture within the roman empire. This work takes place within the EPNET project~\cite{remesal_epnet_2014}.


%XRC this process played a role in the trajectories of cultural change of the different regions, which were interconnected by trade
The trade possibilities between provinces modify the transmission of culture within each province and between them. The influence can comes from two main sources. First, when a new connection is made between two provinces, new products can arrive in their respective markets. The arrival of new goods naturally changes the rate of adoption of every goods of the market CITE. Moreover, the arrival of new products can lead to their recombination with previous products and therefore the creation of an new item. This further modify the structure of the market of the province. Second, persons are also travelling on the roads of the trade network, each carrying its own culture. Large movements of populations may lead to changes in the cultures. For example, when Roman soldiers were stationing in a province, Roman products such as olive oil were delivered to that province CITE ROMAN DELIVERING OIL. This can lead to the adoption of olive oil in regions that were originally not using it.


%XRC these two elements (cultural transmission and trade networks) are traditionally been studied as isolated systems.
Traditionally cultural transmission and trade are studied distinctively.  On the one hand, the cultural transmission mechanisms studied do not take economical consideration into accounts. The works done are interested in the adoption of various cultural trait without taking into consideration their economical value. On the other hand, trading behaviours are studied within their own framework which are designed to measure the amount of goods exchanged and the value at which they are exchanged. Within these frameworks the cultural transmission between persons is not present.


%XRC this is wrong because blablabla
As it has been stated before, trades influence cultural transmission by modifying both the goods present in a region and the people consuming these goods. On the one side, analysing separately trade and cultural transmission allows to formalize their inner mechanisms and understand their respective dynamics. On the other hand, in order to analyse the adoption of goods within a province one must take into account both trading strategy and cultural transmission. For example, the modification of good's consumption after the rupture of a connection in the trade system is better explained by knowing the local adoption of this good (cultural transmission) and the prices of this good (trade).

%XRC we propose a model of coevolution to explore the deep links between trade and cultural transmission in the ancient Mediterranean
We propose in this article a framework to study the interaction between cultural transmission and trade. This framework should be able first to reproduce well known dynamics of both cultural transmission and trading. Second, to be useful, this framework should be able to compare results obtained by taking cultural or trade into consideration. In order to build this framework we view trade events as part of the culture. The price of a good are then viewed as a cultural trait. This traits can be transmitted and modified along different rules. By this mean, traditional works previously expressed in a trading framework, can be expressed in a cultural evolution framework. Therefore, the dynamics observed on various cultural transmission mechanisms can be straightforwardly compared to dynamics obtained by trading mechanisms. In order to demonstrate the feasibility of this approach we propose a minimal implementation able to reproduce both cultural transmission dynamics and trade dynamics. Additionally we will analyse the results obtained with tools from both economy and cultural evolution.




%Organisation of the paper
\todo{write once we are more sure of what we put}
The paper is organised as follow

\section{PREVIOUS MODELLINGS OF TRADE AND CULTURE}

\todo{and/or is ugly, look for another way}
This section will provide a brief review of the previous approaches to model trade and/or culture. While non-exhaustive, this review presents the mains trends of the previous work done.

\subsection{Modelling of Trade}

%Hopkins 1980: structure of tax system. two hypothesis. 1)taxation increased volume of trade in the empire? 2) tax-exporting province (recently integrated province) had to export to pay tax. roman supported by many small taxes. Not sure to cite it

Within the archaeological field a large number of works have been proposed to quantify the properties of the economical interactions in the ancient Mediterranean societies~\cite{hopkins_taxes_1980,temin_market_2001,terpstra_trade_2011,minc_monitoring_2006,temin_economy_2006,wilson_approaches_2009,scheidel_model_2007,kessler_organization_2007}. In the works of~\cite{temin_market_2001,temin_economy_2006,wilson_approaches_2009}, based on the analysis of archaeological evidence, the Roman economy has been classified as a type of market economy where a variety of small tightly connected markets had loose connections between them. This view of a market system in ancient societies has been further expended in~\cite{minc_monitoring_2006} by a model applicable to various cases and studied in a test case of the aztec economical structure. The reaming works attempted at defining more exactly which type market economy it was, i.e. was it close to ours or remote ? 


The work of~\cite{scheidel_model_2007} proposes to analyse the factors affecting the growth of income and population in the Roman Italy. The analysis of historical evidence and historical documents lead the author to propose a model highlighting the key factors in the economical growth. More in link with the work conducted in this article~\cite{terpstra_trade_2011,kessler_organization_2007} have analysed the interactions between economy and culture transmission in roman empire. An analysis based on historical documents in~\cite{kessler_organization_2007} highlights the importance of economical and social institutions in order to perform an efficient trade of trains in the early Roman Empire. More precisely, the work conducted in~\cite{terpstra_trade_2011} proposes a micro-economic model analysing the possibility of trade under uncertainty in the roman empire. This work highlight the importance of reputation and the threat of expulsion from a trade network as mechanisms the overall efficiency of the trade network. 


A limitation of the works cited above is the impossibility to test the conclusions reached by the authors. The models proposed reflects only the view of the authors on the data they have analyse. A new type of model would be necessary to go beyond the interpretation, and allow the verification of the conclusions reached. These models should be able to generate data which is then qualitatively compared to the real historical data. By this mean, if a factor is really important in the economical network, its removal from the model will lead to data qualitatively different from the historical data. The Multi-Agent Simulations (MAS) offer a way to produce this kind of model~\cite{lake_trends_2014}.

Based on this tool a relatively complex and exact modelisation of the prehispanic Pueblo societies has been proposed~\cite{kohler_modelling_2012}. This work offer a relatively exact MAS integrating notable the effects food consumption, production and landscape. The questions studied are  numerous and some works are performed on the economical system and its ties with social transmissions~\cite{kobti_emergence_2006,cockburn_simulating_2013}. However the work done in this project consider a pre-market society in a specific village which makes it difficult to reach more general conclusions, notably applicable to the Roman Empire.


%kobti 2012 la meme que kobti 2006 ?
%	pour 2006: etude de l'apparition d'inegalite


\subsection{Modelisation of Culture}

\xrc{models of cultural change, not culture in general?}

\xrc{what is exactly the definition of culture used in the paper? I would suggest this one: \emph{Culture is socially learned information capable of affecting  individual phenotypes. People acquire culture from other individuals, via teaching or imitation} \cite{richerson_principles_1996}}

The second aspect we wish to model is the evolution of the culture in the Roman empire. The evolutionary framework (generally used in Biology) has been proposed to study the changes of culture and found to be suitable,  particularly in the case of historical and archaeological studies~\cite{lycett_cultural_2015,henrich_evolution_2003}\xrc{henrich's paper is not on archaeology or history, better reference \cite{shennan_evolution_2008}}.

Within this framework, the individual preferences for cultural traits (such as music taste or shapes of pottery) are copied from one individual to the other. Each individual is then adding some small variations to this cultural trait she just acquired. The individual will in turn continue to spread these cultural traits which will be further adopted by other individuals. Studies are then performed on the three key steps of the framework: the transmission, the variation and the differential replication. Interestingly the study of these elements is not only done thanks to humans but also thanks to other animals.

\xrc{I would use social learning instead of cultural transmission}

The transmission mechanisms are the means used to transfer cultural traits from one individuals to another. A large number of social learning schemes can be envisioned~\cite{heyes_social_1994} \xrc{a better reference would be \cite{henrich_evolution_2003}}. Notably, the mechanisms do not necessarily have to be as complex as teaching, since emulation and imitation can be sufficient to pass on certain cultural traits.

The variation mechanisms are at the heart of the creation of the creation of new cultural elements~\cite{obrien_variation_1990}. These may happen due to some errors during the copy process~\cite{schillinger_copying_2014}. In more recent culture we can witness the apparition of mutations with intentionality~\cite{ziman_technological_2003}\xrc{see also recombination \cite{sole_evolutionary_2013}}.

\xrc{Instead of differential replication I would use \emph{biased selection}}
The last element is the one that will interest us the most in this study: the differential replication between different cultural traits. If a cultural trait is found more often than the others in the population it means that something modified the rate at which it is adopted by each agent \xrc{beyond drift effects produced by random copy of neutral traits \cite{bentley_random_2004}}.
%The reproduction rate of ideas in a population can be characterise by different rates: directional, disruptive or stabilisation~\cite{rendell_cognitive_2011}. 
Multiple processes can be at play behind the reproduction \xrc{selection? transmission? reproduction is too biological} of cultural traits. Among them the three following will interest us due to their simplicity to test in an agent-based modelisation. First is the random copy of one trait from an agent to another agent. 
\todo{get a ref on that} \xrc{you can use \cite{bentley_random_2004}}
This type of mechanism reproduces well the changes in musical preferences \xrc{music here is odd; if an example is used it should be related to the topic of the paper}. Second are the the copy based on the frequency at which a cultural trait is observed. If a common trait is reproduced more often then we speak about ``conformity.'' If on the contrary the less common trait is preferred for reproduction we speak of ``rarity'' \xrc{nope, we speak of anti-conformism}. Finally a trait can be preferred depending on whom is carrying it. Among these preferences the prestige of the individual carrying it can constitute an explanation for the adoption foreign cultural traits in the Roman empire \xrc{this is known as model biased transmission}.

\xrc{Again, I should focus the lit.review on models of cultural change in at least ancient societies, this is too broad.}

\subsection{Modelisation of Trade and Culture}

\todo{speak about the fact that economy looks at convergence and culture looks before}

To the best of our knowledge two works have previously proposed frameworks that could be used to model the co-evolution of trade and culture. The model in~\cite{bentley_specialisation_2005} is proposed to study the wealth inequalities in a dynamic network. In this model the agents are in a virtual environment and can produce, consume and trade two products. This model has two major constraints which makes it unfit for the study presented here. First, the products are only for trade purposes and are not used as cultural factors, i.e. some goods can not be accumulated so that the agent show them to others. The cultural aspects appear only through the price associated to each product. Second, only two goods are considered and the system can not scale up to an undetermined number of products without major modifications.
% Agents can negotiate the rate at which the two products will be traded. The agents have a will (hard-coded in their behaviour) to keep their wealth and number of agents within their trading network above a threshold.

The model proposed in~\cite{macmillan_agent-based_2008} studies the settlement of primitive agricultural societies. In this model agents produce, trade and consume 2 essentials goods and 2 non-essentials goods. This model is interesting since it introduce the notion of non-essential goods which can be used to reflect the culture of a society. However, this framework is constrained as it does not allow the evolution of the trading network. Moreover the behaviour chosen for the agents is rather complex and not justified by observations made in economy or evolution of culture.

The model presented in this work proposes to integrate the most interesting features from~\cite{bentley_specialisation_2005} and~\cite{macmillan_agent-based_2008}.

\xrc{here a research question should be defined, and some hypotheses, e.g.: \emph{Did trade networks between regions bias cultural change towards anti-conformism?} or \emph{What is the relation between shipment time and cultural change?}}

\xrc{A co-evolutionary model is really difficult to study, so a careful justification needs to be done on why this is here needed}

\section{MODEL DESCRIPTION}

In the model presented a set of agents are each in possession of a fixed number of goods and associate a value to each of these goods. Depending on the question studied, the value can reflect the interest of the agent for the good or it can be the price at which the agent evaluate this good. The agents are also endowed with limited storage abilities for each good that can be produced. 


More formally the model is composed as follow. $r$ is the number of resources present in the environment. The whole population is defined by a vector of $m$ agents: 
$$ population = (a_i, \cdots ,a_m) $$
		
Each agent $i$ is defined by 2 vectors each of size $r$. The first correspond to the quantity of each good that the agent possesses: $$ quantity = (q^i_1,\cdots,q^i_r) $$

Each agent makes its own estimate on the price to attribute to a certain good. This price estimate is used for both selling and buying goods :
$$ values = (v^i_1,\cdots,p^i_r) $$

On top of these elements two processes are included : imitation and innovation. The first one reflects the possibility for agents to imitate other agents. For this, an agent can copy the value vector of another agent. The second process reflects the possibility for an agent to associate new values for the goods. For this, an agent can select new elements for its value vector independently of the values of other agents. 

This scheme allow us to have a model achieving the two main goals listed in section~\ref{sec:intro}. First, this model allows us to test various cultural transmission mechanisms as well as trade mechanisms. The experiments presented in this article show examples of implementations for both categories. Moreover, the fact that only two processes are modified allow comparisons between simulations with the same units. Second, few parameters are present in the default model which makes it easier to analyse and reuse by other researchers.


\section{EXPERIMENTAL SETUP}


In order to validate our model a first result from the literature on the evolutionary transmission is reproduced. We then illustrate the possibility to implement a trade mechanisms.

\subsection{Cultural Transmission}

In this article the cultural transmission tested is the ``random copy" such as presented in~\cite{mesoudi_random_2009}. The aim is to show that our model is indeed able to reproduce the main trends expected with such copy mechanism. Namely, we should observe a distribution known as the power law, displaying a line in the log-log graph of the proportion of variants with regards to the frequency of variant.

The imitation process is a ``random copy" meaning that with a low probability each agent pick randomly one agent among all and copy it's vector of values. The innovation process is triggered with a low probability and select new elements for the vector of values. These new elements are drawn from a uniform distribution. The imitation and innovation probabilities are presented with other parameters in table~\ref{tab:parameters}


\subsection{Trading Model}\label{sec:trade}

\paragraph*{Imitation}
To test a simple trading model, two notions are introduced in the imitation mechanism: need and score. The need is a quantity of product that each agent tries to obtain. This quantity is different for each product but the need for a product is the same for all agents:
$$ need = (n_1, \cdots, n_r) $$ 

The score reflects the ability for an agent to obtain the products it needs, it is formally computed as follow for agent $i$:

$$ \abs{\dfrac{q^i_j - n_j}{ \sqrt{\abs{(q^i_j*q^i_j)-(n_j*n_j)}}}} $$


\paragraph*{Trading}

On top of producing the agents have the ability to exchange and consume the goods. 
On the one side, the structure of the environment and the proceeding of a simulation will be the same for all hypothesis tested in this model. The specific rules for production, consumption and trading vary depending on the hypothesis tested. Nevertheless 

\paragraph*{Innovation}

One time step of the simulation is divided in three phases. In the first phase all agents produce goods. Each agent is assigned one type of good, and it will produce only this type of good at each first phase. In the second phase all agents enter trade with other agents according to their own interest. Each agent has the possibility to choose with which other agent it will trade products. Once two agents agree to trade, the agents decide which products will be exchanged and in which quantities. When an agreement is found, the trade occur and the agents look for other potential partners. Within one trading phase, an agent can meet only one time each other agents. In the third phase all agent consume goods according to their own rules. A unit of a good consumed disappears from the stock of the agent.

%In the simulation agents produce goods that they will barter (cf section \ref{trade}) and consume (cf section \ref{consumption}). This goods production should reflect the historical good production of the period studied and agent could produce various good in order to barter or sell them using the trade network.




The production methods can vary depending of geographical or historical factors. In order to validate the relevance of the model, we will follow \cite{gintis_emergence_2006}. In this model, the goal is to observe the convergence of price to a market equilibrium in a bartering decentralized economy model. In the orignal model it is shown that the prices can converge to the market clearing price.

In this economical model the agents are ranked with regards to the value of their stock. This value is computed as follow ...
\todo{find the computation}

In this model each agent produces only one type of good. During the trading process, all agents meet each other. Since one agent is producing only one good it will only trade with the agents not producing this good. The procedure to choose the trading partners is detailed in algorithm XXXX.
\todo{do the algorithm}
Briefly summarised, for each good that it does not produced, an agent will look for the parter that offer the best trade, i.e. the agent that proposes to exchange the most of the other product for the fewest of the product produced by the agent.

All agents of the simulation have the same constant consumption vector. A fixed-for-everyone proportion $b_1,\cdots,b_n$ of each resource $(1, \cdots, n)$ is given at the beginning of the simulation. At the end of the consumption phase all remaining products are removed from the agents.

At the end of a time step, each agent has the opportunity to update its prices . Following the strategy proposed in the original model the XXX\% agents with the worst economical results adopt the strategy of the XXX\% best agents.



\section{RESULTS}

\section{CONCLUSION}

\section{DISCUSSION}

%The questions we would like to ultimately study are for example: how does the hierarchical structure changes the adoption of a new product ? How delays and interruptions in the trade network change the culture of a society ?

In future works numerous, more realistic and complex dynamics can be studied in the same model. First the constraints on the production could be relaxed so that each agent can choose the products it wishes to produce at the current time step. Moreover meterological factors can modify the quality of the production (noisy outcome of the proces) thus making the process more realistic and more complex~\cite{bentley_specialisation_2005}.

The goods themselves can be of different type: ``vital'' or ``common''. The absence of a ``vital'' good would then lead to the death of the agent. Thus making the need to trade at the best price even stronger for the agent. The ``common'' goods on the other sides would be interesting to observe the evolution of a culture out of the necessity of survival. The mechanisms behind the strong use of a non-essential product can then be studied. Naturally in the study of these questions the consumption function would change trough time for the agent.

%ACE and prospective trading
More complex models can also be envisionned such as those developed by 
\cite{rubinstein_equilibrium_1985} and to test more general proposal, as those done by \cite{polanyi_trade_1957,polanyi_livelihood_1977} about the early economical mechanisms. Moreover, and in order to be more realistic, and consistently with the cognitive and agent based approach of the project, we will also test assumptions made by people from \emph{Prospect Theory} as proposed by 
\cite{kahneman_prospect_1979}, see also
\cite{camerer_prospect_2004}. The idea here is that traditional economical studies use formal mathematical models that suppose people acting as rational agent, which is a presumption that is far from being valid  in every day life. 

Top palliate this problem, Prospect Theory put the focus on the study of the human cognitive abilities and the impact those abilities brings on economics issues. Among various aspect on which prospect theory shed light one can look at how cognitive abilities modify the decisions-making process of human under risk
\cite{weber_disposition_1998}.

Moreover, we think that this approach fit perfectly with our agent based modeling approach (also called Agent based Computational Economy, ACE, when dealing with economical problems, see
\cite{tesfatsion_introduction_2001}).


This subjective value is the value that agents ``put'' on each resources. It could be learn using any mechanism of ``social learning'' (as defined by \cite{lycett_cultural_2015}) known in the literature (teaching, different kind of copying mechanisms,\ldots) and/or integrate any cognitive/environmental bias that could be studied (see again \cite{lycett_cultural_2015} for some kind of bias that could be implemented).

In order to test these hypohtesis more advanced agents are naturally envisionned. One possible tracks of investigation is agents endowed with self-adaptation abilities as proposed in Reinforcement Learning or Developmental Learning. Another possibility is to follow the cognitive approaches that propose to model the complexity of the human behaviour in general representations such as BDI and XXXX.


\bibliographystyle{wsc}
\bibliography{wsc.bib}  
\end{document}

